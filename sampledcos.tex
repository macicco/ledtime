\documentclass[12pt,a4paper]{article}
\usepackage[utf8]{inputenc}
\usepackage{amsmath}
\usepackage{amsfonts}
\usepackage{amssymb}
%\usepackage{draftwatermark}
%\SetWatermarkScale{4}
%\SetWatermarkLightness{0.9}
\usepackage{fancyhdr}
\pagestyle{fancy}
\fancyhf{}
\author{Michael~Hirsch, MSEE,  Ph.D. EE candidate }
\title{When does a sampled cosinuoid equal 0?}
\rhead{draft} 
\begin{document}
\maketitle

For continuous time $t$, trivially:
\[
y = \cos{t} = \pm 1, \quad t\in 0,\pm \pi, \pm 2\pi,\ldots, \pi m, \quad m\in \mathbb{Z}
\]

Likewise for continous time, a cosinusoid with angular frequency $\omega$ and time $t$:
\[
y = \cos{\omega t} = \pm 1, \quad t\in 0,\frac{\pi}{\omega},\frac{2\pi}{\omega},\ldots, \frac{\pi m}{w}, \quad m\in \mathbb{Z}
\]

By inspection, for continuous time frequency $f$:
\[
y = \cos{2\pi f t} = \pm 1, \quad t\in 0,\frac{\pi}{2 \pi f},\frac{2\pi}{2\pi f},\ldots, \frac{\pi m}{2 \pi f}, \quad m\in \mathbb{Z}
\]


The same result is realized in discrete time with sampling period $T_s$ at time samples $n$:
\[
y = \cos{2\pi f t} = \pm 1, \quad n = \textrm{round}\left[\frac{\pi}{2\pi f T_s}\right] m, \quad m\in \mathbb{Z}
\]
where round[$\cdot$] means round to the nearest integer.
That is, the discrete time output $y=\pm 1$ at the sample indices in the vector $n$.
To account for a phase shift $\phi$, a similar argument leads to the final desired result:

\begin{equation}\label{eq:sampint}
n = \textrm{round}\left[\frac{\pi}{2\pi f T_s} + \left|\frac{\phi}{2\pi f T_s}\right|\right]
\end{equation}

In Matlab, \eqref{eq:sampint} is implemented via the code:

\begin{verbatim}
sampinterval = pi./(2*pi*f*Ts);
sampoffset = abs(phase./(2*pi*f*Ts)) + 1 
n = round(sampinterval:sampoffset:ns)
\end{verbatim}
where the \texttt{+1} accounts for the one-based Matlab indexing, and \texttt{ns} is the total number of samples.
\end{document}