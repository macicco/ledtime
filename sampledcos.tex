\documentclass[12pt,a4paper]{article}
\usepackage[utf8]{inputenc}
\usepackage{amsmath}
\usepackage{amsfonts}
\usepackage{amssymb}
\usepackage{draftwatermark}
\SetWatermarkScale{4}
\author{Michael~Hirsch, MSEE }
\title{When does a sampled cosinuoid equal 0?}
\begin{document}
\maketitle

For continuous time $t$, trivially:
\[
y = \cos{t} = \pm 1, \quad t\in 0,\pm \pi, \pm 2\pi,\ldots, \pi m, \quad m\in \mathbb{Z}
\]

Likewise for continous time, a cosinusoid with angular frequency $\omega$ and time $t$:
\[
y = \cos{\omega t} = \pm 1, \quad t\in 0,\frac{\pi}{\omega},\frac{2\pi}{\omega},\ldots, \frac{\pi m}{w}, \quad m\in \mathbb{Z}
\]

By inspection, for continuous time frequency $f$:
\[
y = \cos{2\pi f t} = \pm 1, \quad t\in 0,\frac{\pi}{2 \pi f},\frac{2\pi}{2\pi f},\ldots, \frac{\pi m}{2 \pi f}, \quad m\in \mathbb{Z}
\]


The same result is realized in discrete time with sampling period $T_s$ at time samples $n$:
\[
y = \cos{2\pi f t} = \pm 1, \quad n = \textrm{nint}\left[\frac{\pi}{2\pi f T_s}\right] m, \quad m\in \mathbb{Z}
\]
where nint[$\cdot$] means round to the nearest integer.

That is, the output $y=\pm 1$ in discrete time every $n$ samples.

To account for a phase shift $\phi$, a similar argument leads to the final desired result:

\[
n = 
\]
\end{document}